% Options for packages loaded elsewhere
\PassOptionsToPackage{unicode}{hyperref}
\PassOptionsToPackage{hyphens}{url}
%
\documentclass[
]{article}
\usepackage{amsmath,amssymb}
\usepackage{lmodern}
\usepackage{iftex}
\ifPDFTeX
  \usepackage[T1]{fontenc}
  \usepackage[utf8]{inputenc}
  \usepackage{textcomp} % provide euro and other symbols
\else % if luatex or xetex
  \usepackage{unicode-math}
  \defaultfontfeatures{Scale=MatchLowercase}
  \defaultfontfeatures[\rmfamily]{Ligatures=TeX,Scale=1}
\fi
% Use upquote if available, for straight quotes in verbatim environments
\IfFileExists{upquote.sty}{\usepackage{upquote}}{}
\IfFileExists{microtype.sty}{% use microtype if available
  \usepackage[]{microtype}
  \UseMicrotypeSet[protrusion]{basicmath} % disable protrusion for tt fonts
}{}
\makeatletter
\@ifundefined{KOMAClassName}{% if non-KOMA class
  \IfFileExists{parskip.sty}{%
    \usepackage{parskip}
  }{% else
    \setlength{\parindent}{0pt}
    \setlength{\parskip}{6pt plus 2pt minus 1pt}}
}{% if KOMA class
  \KOMAoptions{parskip=half}}
\makeatother
\usepackage{xcolor}
\IfFileExists{xurl.sty}{\usepackage{xurl}}{} % add URL line breaks if available
\IfFileExists{bookmark.sty}{\usepackage{bookmark}}{\usepackage{hyperref}}
\hypersetup{
  pdftitle={Introduction to habCluster},
  hidelinks,
  pdfcreator={LaTeX via pandoc}}
\urlstyle{same} % disable monospaced font for URLs
\usepackage[margin=1in]{geometry}
\usepackage{color}
\usepackage{fancyvrb}
\newcommand{\VerbBar}{|}
\newcommand{\VERB}{\Verb[commandchars=\\\{\}]}
\DefineVerbatimEnvironment{Highlighting}{Verbatim}{commandchars=\\\{\}}
% Add ',fontsize=\small' for more characters per line
\usepackage{framed}
\definecolor{shadecolor}{RGB}{248,248,248}
\newenvironment{Shaded}{\begin{snugshade}}{\end{snugshade}}
\newcommand{\AlertTok}[1]{\textcolor[rgb]{0.94,0.16,0.16}{#1}}
\newcommand{\AnnotationTok}[1]{\textcolor[rgb]{0.56,0.35,0.01}{\textbf{\textit{#1}}}}
\newcommand{\AttributeTok}[1]{\textcolor[rgb]{0.77,0.63,0.00}{#1}}
\newcommand{\BaseNTok}[1]{\textcolor[rgb]{0.00,0.00,0.81}{#1}}
\newcommand{\BuiltInTok}[1]{#1}
\newcommand{\CharTok}[1]{\textcolor[rgb]{0.31,0.60,0.02}{#1}}
\newcommand{\CommentTok}[1]{\textcolor[rgb]{0.56,0.35,0.01}{\textit{#1}}}
\newcommand{\CommentVarTok}[1]{\textcolor[rgb]{0.56,0.35,0.01}{\textbf{\textit{#1}}}}
\newcommand{\ConstantTok}[1]{\textcolor[rgb]{0.00,0.00,0.00}{#1}}
\newcommand{\ControlFlowTok}[1]{\textcolor[rgb]{0.13,0.29,0.53}{\textbf{#1}}}
\newcommand{\DataTypeTok}[1]{\textcolor[rgb]{0.13,0.29,0.53}{#1}}
\newcommand{\DecValTok}[1]{\textcolor[rgb]{0.00,0.00,0.81}{#1}}
\newcommand{\DocumentationTok}[1]{\textcolor[rgb]{0.56,0.35,0.01}{\textbf{\textit{#1}}}}
\newcommand{\ErrorTok}[1]{\textcolor[rgb]{0.64,0.00,0.00}{\textbf{#1}}}
\newcommand{\ExtensionTok}[1]{#1}
\newcommand{\FloatTok}[1]{\textcolor[rgb]{0.00,0.00,0.81}{#1}}
\newcommand{\FunctionTok}[1]{\textcolor[rgb]{0.00,0.00,0.00}{#1}}
\newcommand{\ImportTok}[1]{#1}
\newcommand{\InformationTok}[1]{\textcolor[rgb]{0.56,0.35,0.01}{\textbf{\textit{#1}}}}
\newcommand{\KeywordTok}[1]{\textcolor[rgb]{0.13,0.29,0.53}{\textbf{#1}}}
\newcommand{\NormalTok}[1]{#1}
\newcommand{\OperatorTok}[1]{\textcolor[rgb]{0.81,0.36,0.00}{\textbf{#1}}}
\newcommand{\OtherTok}[1]{\textcolor[rgb]{0.56,0.35,0.01}{#1}}
\newcommand{\PreprocessorTok}[1]{\textcolor[rgb]{0.56,0.35,0.01}{\textit{#1}}}
\newcommand{\RegionMarkerTok}[1]{#1}
\newcommand{\SpecialCharTok}[1]{\textcolor[rgb]{0.00,0.00,0.00}{#1}}
\newcommand{\SpecialStringTok}[1]{\textcolor[rgb]{0.31,0.60,0.02}{#1}}
\newcommand{\StringTok}[1]{\textcolor[rgb]{0.31,0.60,0.02}{#1}}
\newcommand{\VariableTok}[1]{\textcolor[rgb]{0.00,0.00,0.00}{#1}}
\newcommand{\VerbatimStringTok}[1]{\textcolor[rgb]{0.31,0.60,0.02}{#1}}
\newcommand{\WarningTok}[1]{\textcolor[rgb]{0.56,0.35,0.01}{\textbf{\textit{#1}}}}
\usepackage{graphicx}
\makeatletter
\def\maxwidth{\ifdim\Gin@nat@width>\linewidth\linewidth\else\Gin@nat@width\fi}
\def\maxheight{\ifdim\Gin@nat@height>\textheight\textheight\else\Gin@nat@height\fi}
\makeatother
% Scale images if necessary, so that they will not overflow the page
% margins by default, and it is still possible to overwrite the defaults
% using explicit options in \includegraphics[width, height, ...]{}
\setkeys{Gin}{width=\maxwidth,height=\maxheight,keepaspectratio}
% Set default figure placement to htbp
\makeatletter
\def\fps@figure{htbp}
\makeatother
\setlength{\emergencystretch}{3em} % prevent overfull lines
\providecommand{\tightlist}{%
  \setlength{\itemsep}{0pt}\setlength{\parskip}{0pt}}
\setcounter{secnumdepth}{-\maxdimen} % remove section numbering
\ifLuaTeX
  \usepackage{selnolig}  % disable illegal ligatures
\fi

\title{Introduction to habCluster}
\author{}
\date{\vspace{-2.5em}}

\begin{document}
\maketitle

\hypertarget{example}{%
\subsection{Example}\label{example}}

This is a basic example which shows you how to find the cluster of
lands:

\begin{Shaded}
\begin{Highlighting}[]
\FunctionTok{library}\NormalTok{(raster)}
\FunctionTok{library}\NormalTok{(habCluster)}
\end{Highlighting}
\end{Shaded}

Read in habitat suitability index (HSI) data of wolf in Europe. The HSI
values of the cells in the raster indicate how smoothly the wolfs can
moved in the cells, and can be used to represent the connection between
cells as habitat.

\begin{Shaded}
\begin{Highlighting}[]
\NormalTok{hsi.file }\OtherTok{=} \FunctionTok{system.file}\NormalTok{(}\StringTok{"extdata"}\NormalTok{,}\StringTok{"wolf3\_int.tif"}\NormalTok{,}\AttributeTok{package=}\StringTok{"habCluster"}\NormalTok{)}
\NormalTok{wolf }\OtherTok{=} \FunctionTok{raster}\NormalTok{(hsi.file)}
\end{Highlighting}
\end{Shaded}

Find habitat cluster using Leiden Algorithm. Raster for habitat
suitability will be resampled to 40 km (40000m), to reduce calculation
amount. Set cluster\_resolution\_parameter to 0.02 to control the
cluster size.

\begin{Shaded}
\begin{Highlighting}[]
\NormalTok{clst }\OtherTok{=} \FunctionTok{cluster}\NormalTok{(wolf, }\AttributeTok{method =}\NormalTok{ cluster\_leiden, }\AttributeTok{res =} \DecValTok{40000}\NormalTok{, }\AttributeTok{rp =} \FloatTok{0.02}\NormalTok{, }\AttributeTok{silent =} \ConstantTok{FALSE}\NormalTok{)}
\CommentTok{\#\textgreater{} }
\CommentTok{\#\textgreater{} resampling...}
\CommentTok{\#\textgreater{} extracting edges...}
\CommentTok{\#\textgreater{} create graph...}
\CommentTok{\#\textgreater{} finding clusters...}
\CommentTok{\#\textgreater{} preparing results...}
\end{Highlighting}
\end{Shaded}

You can also embed plots, for example:

\begin{Shaded}
\begin{Highlighting}[]
\FunctionTok{image}\NormalTok{(wolf, }\AttributeTok{col =} \FunctionTok{terrain.colors}\NormalTok{(}\DecValTok{100}\NormalTok{,}\AttributeTok{rev =}\NormalTok{ T), }\AttributeTok{asp =} \DecValTok{1}\NormalTok{)}
\FunctionTok{plot}\NormalTok{(clst}\SpecialCharTok{$}\NormalTok{boundary, }\AttributeTok{add =}\NormalTok{ T, }\AttributeTok{asp =} \DecValTok{1}\NormalTok{, }\AttributeTok{border =} \StringTok{"lightseagreen"}\NormalTok{)}
\end{Highlighting}
\end{Shaded}

\includegraphics{/home/qiang/habCluster/habCluster/vignettes/Introduction-to-habCluster_files/figure-latex/cluster-1.pdf}

\hypertarget{how-to-cite}{%
\subsection{How to Cite}\label{how-to-cite}}

Zhang, C., Q. Dai*, et al, (in review). Identifying Geographical
Boundary among Intraspecific Units Using Community Detection Algorithm.

\end{document}
